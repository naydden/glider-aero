\chapter{Ala i estabilitzadors}

Un cop estudiat el comportament de l'ala aïllada, s'ha d'estudiar el comportament del planejador complet, tenint en compte tant l'estabilitzador horitzontal com el vertical i per tant les interaccions i interferències que aixo produeix respecte l'ala aïllada. 

Per tal d'estudiar el conjunt, s'assumeix que l'estabilitzador horitzontal te un angle de torsió o twist nul. A més, s'estableix que l'estabilitzador horitzontal només produeix una resistència aerodinàmica parasita ja que presenta un angle de lliscament $\beta=0^{\circ}$ i per tant no genera sustentació ni conseqüentment resistència induïda. 

\section{Coeficients Aerodinàmics}
 
\section{Posició del centre de masses} 
