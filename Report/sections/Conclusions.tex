\chapter{Conclusions}

S'ha escollit resoldre el problema amb el mètode del \textit{vortex lattice}, doncs suposava una novetat interessant de l'assignatura. A més, per geometríes anormals, o per ales amb un allargament petit, és un mètode més precís que el del \textit{lifting line}. 

El codi emprat per presentar els resultats d'aquest informe, ha estat validat amb els resultats que s'han publicat pel professor Oriol Lizandra, com també amb el programari lliure XFLR5, que també implementa un \textit{vortex lattice}.

Des del punt de vista aerodinàmic, ha estat interessant poder triar un angle de torsió basant-se en el fregament generat, així com veure l'efecte de l'allargament sobre el $C_L$ i el $C_D$. No obstant, la part més enriquidora ha estat veure com influeix l'efecte terra en el comportament de l'avió, validant les hipòtesis inicials comentades al llarg de l'informe.

Des del punt de vista de la programació, s'ha realitzat un treball en equip molt satisfactori, dividint el problema global en petites funcions amb una responsabilitat limitada. Fent això, aquestes funcions s'han anat validant de forma que un cop el programa ha crescut considerablement, existia la confiança en el seu correcte funcionament.

Les possibles millores de cara al futur es centrarien en millorar l'eficiència del codi, reduint així el cost computacional. També seria interessant substituir l'actual model d'estela rígida per un d'estela flexible.
